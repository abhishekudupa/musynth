\documentclass[10pt]{article}
\usepackage{fullpage}
\usepackage{setspace}
\usepackage[fleqn]{amsmath}
\usepackage{amssymb}
\usepackage{mdwtab}
\usepackage{mathenv}
\usepackage{xspace}
\usepackage[usenames,dvipsnames]{color}
\usepackage{natbib}

\newcommand{\ctl}{\textsc{ctl}\xspace}
\newcommand{\ptrans}{\ensuremath{\mathbf{R}_p}\xspace}
\newcommand{\ctrans}{\ensuremath{\mathbf{R}}\xspace}
\newcommand{\ag}[1]{\ensuremath{\mathsf{AG}#1\xspace}}
\newcommand{\eg}[1]{\ensuremath{\mathsf{EG}#1\xspace}}
\newcommand{\ax}[1]{\ensuremath{\mathsf{AX}#1\xspace}}
\newcommand{\ex}[1]{\ensuremath{\mathsf{EX}#1\xspace}}
\newcommand{\au}[2]{\ensuremath{\mathsf{A}#1 \mathsf{U} #2\xspace}}
\newcommand{\eu}[2]{\ensuremath{\mathsf{E}#1 \mathsf{U} #2\xspace}}
\newcommand{\pgood}{\ensuremath{P_{\mathrm{good}}}\xspace}
\newcommand{\post}[2]{\ensuremath{\mathsf{img}(#1, #2)}\xspace}
\newcommand{\pre}[2]{\ensuremath{\mathsf{pre}(#1, #2)}\xspace}
\newcommand{\todo}[1]{\paragraph{\textcolor{Red}{\textbf{TODO:}}} #1}
\newcommand{\lsub}[2]{\ensuremath{\mathsf{lsub}(#1, #2)}\xspace}


\title{A Symbolic Algorithm for Automata Completion with \ctl
  Properties}
\author{Abhishek Udupa}
\date{}

\begin{document}
\begin{spacing}{1}
\maketitle

\section{Preliminaries}
We consider the problem of completing a set of IO automata such that the
completion satisfies the given \ctl specification. The incomplete (product)
system is assumed to be provided as a single initial state $s_0$ and a
transition relation $\ctrans \subseteq Q \times \Sigma \times Q$, where $Q$
is the set of product states and $\Sigma$ is the alphabet of the product
automaton. Note that the transition relation $\ctrans$ is constructed using
the standard notion of product of IO automata from the transition relations
of the individual automata which are assumed to be known.

For each state, alphabet pair $\langle q, \sigma\rangle$ in each individual
IO automaton, such that a transition on input/output $\sigma$ in state $q$
is unspecified, we introduce a parameter $t_{q,\sigma}$ whose value ranges
over the set of states of the automaton under consideration. We denote the
set of all parameter variables by $T$ and let $P$ be the set of
\emph{valuations} $T$. We thus obtain a \emph{parametrized} transition
relation $\ptrans \subseteq Q \times P \times \Sigma \times Q$. The
objective is to find a valuation $p \in P$ such that the (concrete)
transition relation $\ptrans[T := p]$ obtained by replacing each parameter
in $T$ with its valuation from $p$ satisfies a given \ctl specification
$\varphi$.

A straightforward algorithm to solve this problem would be to allow the
parameters to take on any value from their respective domains in the
initial state. The single (concrete) initial state $s_0$ of the transition
system will now result in a set of initial states, denoted by $s_0^p$, with
one concrete state for each valuation of parameters, which can be
represented symbolically. We can then simply check if any state in $s_0^p$
satisfies the $\varphi$ by standard symbolic model checking algorithm.

The problem with this straightforward algorithm is that they proceed using
a global fixpoint formulation driven by the structure of $\varphi$. This
can result in a large number of states which are unreachable being
examined. The presence of parameters only exacerbates this problem. An
alternative approach, which is presented here, attempts to avoid this
problem by using a \emph{forward} analysis, in which only reachable states
are considered and valuations of the parameters which can be certified as
``bad'' are eagerly, but conservatively, pruned.

We present an algorithm which given a set of states $S$ and a \ctl formula
$\varphi$, produces the largest subset $S' \subseteq S$, such that for
every $s \in S'$, $s \models \varphi$. The algorithm is inspired by the
forward \ctl model checking algorithm by Iwashita,
et. al.~\cite{ctl-forward}. Such a procedure could then be trivially used
to solve the synthesis problem, by finding the largest subset of the set of
(parametrized) initial states $s_0^p$.

We use the following notations in the presentation: \lsub{S}{\varphi}
refers to the largest subset of $S' \subseteq S$ such that for every state
$s \in S'$, $s \models \varphi$.

\section{The \eu{}{} Operator}
Suppose we are given a set $S$ and a \ctl formula of the form
\eu{\psi_1}{\psi_2}, observe that the fixpoint
$$t_1 = \mu Z. S \wedge \post{\lsub{Z}{\psi_1}}{\ctrans}$$
consists of states $t$ such that $t$ is reachable from some state in $S$
through a path along which $\psi_1$ holds at least until the predecessor of
$t$. Now,
$$t_2 = \lsub{t_1}{\psi_2}$$ gives us precisely the set of states $s$ such
that $s \models \eu{\psi_1}{\psi_2}$

\section{The \eg{} Operator}
Suppose the \ctl formula has the form $\eg{\psi}$, then we compute
$$t_1 = \lsub{S}{\eu{\psi}{\psi}}$$,
which is the set of all $\psi$ states that can be reached from some state
in $S$

\bibliographystyle{plain}
\bibliography{alg01}

\end{spacing}
\end{document}
