\documentclass[10pt]{article}
\usepackage{fullpage}
\usepackage{setspace}
\usepackage[fleqn]{amsmath}
\usepackage{amssymb}
\usepackage{mdwtab}
\usepackage{mathenv}
\usepackage{xspace}

\newcommand{\ctl}{\textsc{ctl}\xspace}
\newcommand{\ptrans}{\ensuremath{\mathbf{R}_p}\xspace}
\newcommand{\ctrans}{\ensuremath{\mathbf{R}}\xspace}
\newcommand{\ag}[1]{\ensuremath{\mathsf{AG}#1\xspace}}
\newcommand{\eg}[1]{\ensuremath{\mathsf{EG}#1\xspace}}
\newcommand{\ax}[1]{\ensuremath{\mathsf{AX}#1\xspace}}
\newcommand{\ex}[1]{\ensuremath{\mathsf{EX}#1\xspace}}
\newcommand{\au}[2]{\ensuremath{\mathsf{A}#1 \mathsf{U} #2\xspace}}
\newcommand{\eu}[2]{\ensuremath{\mathsf{E}#1 \mathsf{U} #2\xspace}}
\newcommand{\pgood}{\ensuremath{P_{\mathrm{good}}}\xspace}
\newcommand{\post}[2]{\ensuremath{\mathsf{post}(#1, #2)}\xspace}
\newcommand{\pre}[2]{\ensuremath{\mathsf{pre}(#1, #2)}\xspace}


\title{A Symbolic Algorithm for Automata Completion with \ctl
  Properties}
\author{Abhishek Udupa}
\date{}

\begin{document}
\begin{spacing}{1}
\maketitle

\section{Preliminaries}
We consider the problem of completing a set of IO automata such that the
completion satisfies the given \ctl specification. The incomplete (product)
system is assumed to be provided as a single initial state $s_0$ and a
transition relation $\ctrans \subseteq Q \times \Sigma \times Q$, where $Q$
is the set of product states and $\Sigma$ is the alphabet of the product
automaton. Note that the transition relation $\ctrans$ is constructed using
the standard notion of product of IO automata from the transition relations
of the individual automata which are assumed to be known.

For each state, alphabet pair $\langle q, \sigma\rangle$ in each individual
IO automaton, such that a transition on input/output $\sigma$ in state $q$
is unspecified, we introduce a parameter $t_{q,\sigma}$ whose value ranges
over the set of states of the automaton under consideration. We denote the
set of all parameter variables by $T$ and let $P$ be the set of
\emph{valuations} $T$. We thus obtain a \emph{parametrized} transition
relation $\ptrans \subseteq Q \times P \times \Sigma \times Q$. The
objective is to find a valuation $p \in P$ such that the (concrete)
transition relation $\ptrans[T := p]$ obtained by replacing each parameter
in $T$ with its valuation from $p$ satisfies a given \ctl specification
$\varphi$.

A straightforward algorithm to solve this problem would be to allow the
parameters to take on any value from their respective domains in the
initial state. The single (concrete) initial state $s_0$ of the transition
system will now result in a set of initial states, denoted by $s_0^p$, with
one concrete state for each valuation of parameters, which can be
represented symbolically. We can then simply check if any state in $s_0^p$
satisfies the $\varphi$ by standard symbolic model checking algorithm.

The problem with this straightforward algorithm is that they proceed using
a global fixpoint formulation driven by the structure of $\varphi$. This
can result in a large number of states which are unreachable being
examined. The presence of parameters only exacerbates this problem. An
alternative approach, which is presented here, attempts to avoid this
problem by using a \emph{forward} analysis, in which only reachable states
are considered and valuations of the parameters which can be certified as
``bad'' are eagerly, but conservatively, pruned.

We present algorithms for the \ctl operators \ag{}, \eg{}, \au{}{},
\eu{}{}, \ax{} and \ex{}. We also assume that negations are applied only to
propositional formulas. Note that even with this restriction, we can still
express any \ctl formula with the vocabulary of operators we consider.

\section{The \eg{} Operator}
Suppose that the specification is of the form \eg{\psi}. We first describe
an algorithm to determine if a given set of states $S$ satisfies
\eg{\psi}, with respect to a fixed (concrete) transition relation
$\ctrans$, which we then extend to obtain the set of all valuations $\pgood
\subseteq P$ which result in the parametric transition relation $\ptrans$
satisfying the specification $\eg{\psi}$.

\subsection{Model Checking for $\eg{\psi}$}
Observe that a given set of states $S \models \eg{\psi}$ if and only if at
least one state $s \in S \models \psi$ and there exists an infinite path
beginning at $s$ where $\psi$ holds everywhere. To state this formally, we
use the notation $\post{S}{\ctrans}$ to denote the set of states reachable
from any state $s \in S$ in one step with respect to the transition
relation \ctrans. We abuse notation and use $\post{s}{\ctrans}$ to refer to
the set of states reachable from the single state $s$ in one step with
respect to the \ctrans. Similarly, we use the notation $\pre{S}{\ctrans}$
to refer to the set of states which can reach a state in $S$ in one step
with respect to $\ctrans$. We then obtain the following sequence of
equivalences:

\begin{eqnarray*}[ll]
& \ S \models \eg{\psi}\\
\Leftrightarrow & \ S \models \psi \wedge (\exists s' \in
\post{S}{\ctrans}. s' \models \eg{\psi})\\
\Leftrightarrow & \ S \models \psi \wedge (\exists s' \in
\post{S}{\ctrans}. s' \models \psi \wedge (\exists s'' \in
\post{s'}{\ctrans}. s'' \models \eg{\psi}))
\end{eqnarray*}
and so on. This sequence of equivalences suggests the following technique
to determine if a given set of states $S \models \eg{\psi}$: We construct a
sequence of sets of states $F_i, i \geq 1$ such that each state in $F_i$ is (1)
reachable from some state in $s \in S$ in $i$ steps, and (2)
each state $s \in F_i \models \eg{\psi}$. The set $\bigcup_i F_i$ thus
gives the set of \emph{all} states reachable from $S$ which satisfy
$\eg{\psi}$. There are two questions that need to be answered to construct
an algorithm that implements the technique described: (1) When do we
terminate? (2) How do we construct the sets $F_i$? For the first question,
we terminate either when (1) some $F_i$ becomes empty, implying that the
property is false, or (2) when a fixpoint is reached, i.e., $F_i = F_{i+1}$
for some $i$, implying that the property is true. For the second question,
instead of maintaining the sets $F_i$ exactly, we construct a sequence of
\emph{over-approximations} $F_i^0, F_i^1, \ldots,$, such that $F_i^j$
represents the set of states which are reachable from some state $s \in S$
in $i$ steps \emph{and} satisfies the property \eg{\psi} for \emph{at
  least} $j$ more steps. Intuitively, we construct the following table with
arrows indicating the dependencies between constructions, and where each
entry is a set of states:

\begin{eqnarray*}[lclclclclcl]
& & U_1 & & U_2 & & U_3 & & U_4 & & \ldots \\
& \nearrow & & \nearrow & & \nearrow & &\nearrow \\
F_1^0 & \rightarrow & F_1^1 & \rightarrow & F_1^2 & \rightarrow & F_1^3 &
\rightarrow & \ldots\\
\downarrow & \nearrow & & \nearrow & &\nearrow&\\
F_2^0 & \rightarrow & F_2^1 & \rightarrow & F_2^2 & & \rightarrow & \ldots &\\
\downarrow & \nearrow & & \nearrow & & &\\
F_3^0 & \rightarrow & F_3^1 & \rightarrow & \ldots & & & &\\
\downarrow & \nearrow & & & & &\\
F_4^0 & \rightarrow & \ldots & & & & &\\
\vdots
\end{eqnarray*}
Assuming that all states $s \in S \models \psi$, we construct the sets
$F_i^j$ and $U_i$ using the following recurrences:
\begin{eqnarray*}[lcll]
F_1^0 & = & \post{S}{\ctrans} \wedge \psi &\\
F_i^0 & = & \post{F_{i-1}^0}{\ctrans} \wedge \psi, & \ i > 1\\
F_i^j & = & F_i^{j-1} \wedge \pre{F_{i+1}^{j-1}}{\ctrans}, & \ i > 1,\ j >
1\\
U_i & = & \bigcup_{k \in [0,i-1]} F_{i-k}^{k}, &\ i > 0
\end{eqnarray*}
The sets $U_i$ are essentially \emph{upper bounds} of the sets of states
reachable in $i$ or fewer steps from $S$ which also satisfy
$\eg{\psi}$. The fixpoint where $U_k = U_{k+1}$ is the set of \emph{all}
states reachable from $S$ which satisfy \eg{\psi}. On the other hand, if
any set $F_i^j$ becomes empty during the construction, then $S \not\models
\eg{\psi}$.

\subsection{Synthesizing for $\eg{\psi}$}

\end{spacing}
\end{document}
